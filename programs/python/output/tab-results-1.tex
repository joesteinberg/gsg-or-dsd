\begin{table}[p]
\footnotesize
\begin{center}
\begin{threeparttable}\caption{Wedge accounting results for U.S. quantities and prices}
\label{tab:tb-results-1}
\begin{tabular}{lccccccc}
\toprule
& & & \multicolumn{2}{c}{United States} & \multicolumn{3}{c}{Rest of the world}\\
\cmidrule(rl){4-5}\cmidrule(rl){6-8}
Measure & Data & No wedges & Saving wedge & Inv. wedge & Saving wedge & Inv. wedge & Trade wedge\\
\midrule
\multicolumn{8}{l}{\textit{(a) Trade balance (percent GDP)}}\\
Minimum& -5.10 & 3.87 & 3.07 & 4.15 & -4.33 & 2.55 & -0.61\\
Average& -3.20 & 4.86 & 4.58 & 5.32 & -2.90 & 4.46 & 5.14\\
RMSE from data& 0.00 & 8.21 & 7.89 & 8.73 & 1.90 & 7.80 & 8.86\\
Fraction CED explained& 1.00 & 0.00 & 0.03 & -0.06 & 0.96 & 0.05 & -0.03\\
\\
\multicolumn{8}{l}{\textit{(b) Real exchange rate (1995 data = 100)}}\\
Minimum& 79.97 & 116.02 & 112.20 & 115.14 & 86.55 & 117.95 & 116.03\\
Average& 91.92 & 125.35 & 124.67 & 125.83 & 92.02 & 125.40 & 124.19\\
RMSE from data& 0.00 & 34.82 & 34.59 & 35.47 & 4.63 & 34.15 & 33.19\\
Fraction CEA explained& 1.00 & 0.00 & 0.02 & -0.01 & 1.00 & -0.00 & 0.03\\
\\
\multicolumn{8}{l}{\textit{(c) Investment rate (percent GDP)}}\\
RMSE from data& 0.00 & 1.90 & 1.16 & 2.37 & 2.63 & 2.05 & 2.83\\
Fraction CEI explained (1995--2006)& 1.00 & 0.00 & 0.37 & -0.48 & 1.26 & 0.19 & -0.51\\
Fraction CEI explained (2007--2011)& 1.00 & 0.00 & 0.64 & 2.25 & -1.48 & -0.02 & -0.77\\
\\
\multicolumn{8}{l}{\textit{(d) Real interest rate (percent per year)}}\\
Minimum& 1.13 & 1.96 & 1.51 & 1.39 & 1.78 & 1.79 & 0.02\\
Average& 2.60 & 2.81 & 2.57 & 2.74 & 2.80 & 2.92 & 2.85\\
RMSE from data& 0.00 & 1.23 & 1.25 & 0.97 & 0.80 & 0.92 & 2.01\\
Fraction decline explained& 1.00 & -0.46 & -0.73 & 0.47 & 0.17 & -0.04 & -0.69\\
\bottomrule
\end{tabular}
\begin{tablenotes}
\item Notes: The second column reports counterfactual model outcomes when all wedges are set to one. Columns 3--7 report counterfactual model outcomes with one wedge set to its calibrated value in each period and all other wedges held constant. Fraction of CED explained calculated as (cumulative difference between trade balance in model and no wedge counterfactual) divided by (cumulative difference between trade balance in data and no-wedge counterfactual). Fraction of cumulative excess RER appreciation (CEA) and cumulative excess investment (CEI) computed analogously.\end{tablenotes}
\end{threeparttable}
\end{center}
\normalsize
\end{table}
